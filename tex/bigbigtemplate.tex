\documentclass{article}
\usepackage[utf8]{inputenc}
\usepackage[russian]{babel}
\usepackage{url}
\usepackage{setspace,amsmath}
\begin{document}
 
\begin{center}
	{\Huge Краткое описание алгоритма расчета орудийных таблиц}
\end{center}
 
 \section{Введение}
В процессе выбора метода расчет рассматривались два варианта: Решать обратную задачу восстановления исходного угла наводки по известной точке попадания. Но в этом случае решение будет неким действительным числом. В то же время для реального орудия есть конкретный шаг, с которым оно поднимается и опускается. Также у нас в игре расстояния считаются только как целые значения в кабельтовых. Поэтому нет необходимости решать задачу для произвольных координат. Поэтому был выбран другой метод. 

Согласно ему сначала рассчитываются все возможные траектории для различных углов наводки от -90 до +90 градусов с неким шагом ( по умолчанию 0.5 градуса). Следует сразу отметить, что задач решается двумерная, так как угол поворота по горизонтали для расчета баллистики нас не интересует. Поэтому есть только вертикаль и расстояние до цели. В этой плоскости решается задача: найти ту траекторию, которая проходит достаточно близко от, отстоящей о начала координат(точки, где запускается снаряд) на заданное число кабельтовых и находящейся выше или ниже на другое, тоже заданное число кабельтовых. "достаточно близко" - значит, что рассчитывается расстояние от траектории до точки. 
Все данные точки удобно записать в таблицу ( а точнее в двумерные массивы) и для каждой точки сделать перебор всех траекторий.

\section{Уравнения и метод расчета траекторий}

Исходно взяты отсюда \url{https://glebgrenkin.blogspot.com/2014/03/blog-post.html}
В векторном виде:
\begin{eqnarray}
	m\frac{d{\vec v} }{dt}= m{\vec g}- k |{\vec v}|{\vec v}
\end{eqnarray}
Или в координатном виде.
\begin{eqnarray}
	\frac{dv_x}{dt}=-\frac{k}{m}\sqrt{v_x^2+v_y^2}\cdot v_x \\
	\frac{dv_y}{dt}=-g-\frac{k}{m}\sqrt{v_x^2+v_y^2}\cdot v_y	
\end{eqnarray}
Численный метод самый простой - явный первого порядка
\begin{equation}
	\frac{dv}{dt} \approx \frac{v^{t+1}-v^t}{\Delta t}
\end{equation}
Или для данной системы
\begin{eqnarray}
	\frac{v^{t+1}_x-v^t_x}{\Delta t}=-\frac{k}{m}\sqrt{(v^t_x)^2+(v^t_y)^2}\cdot v^t_x \\
	\frac{v^{t+1}_y-v^t_y}{\Delta t}=-g-\frac{k}{m}\sqrt{(v^t_x)^2+(v^t_y)^2}\cdot v^t_y	
\end{eqnarray}
Шаг пришлось взять маленький. $10^{-4}-10^{-5}$ секунды.
$m$- Масса снаряда. $k$- некое "сопротивление воздуха снаряду" - то есть по сути подгоночный параметр. для разумных значений получалось 0.005-0.009 для разных снарядов. Надо подгонять и смотреть.
В процессе расчета каждый шаг --- записывается как точка, в которой побывал снаряд
\section{Выбор траектории}

После того, как траектории рассчитаны --- берутся все точки из заданного диапазона ( в длину от 0 до max, в выотсу от нижней границы до верхней) и для каждой точки с целочисленным расстоянием - перебираются все траектории ( на самом деле не все, об ограничениях позже). Для каждой траектории перебираются все ее точки и для каждой ищется расстояние от точки траектории до заданной точки в пространстве. Если наименьшее расстояние больше некой величины ( ща - один кабельтов) то считается, что нету траекторий и в данную точку орудие попасть не может. Если наименьшее расстояние меньше нужного порога --- то траектория с наименьшим расстоянием считается той траекторией, по которой следует стрелять, чтоб попасть в точку. 
Ограничения: если грубо прикинуть угол межу горизонталью и точкой - то для перебора можно брать только бОльшие углы. Таже не надо брать слишком большой угол ( вроде бы отличающийся больше чем $\pi/4$)Так как стрельба ведется настильная, а не навесная.

Найдя для всех возможных точек свои траектории, можно легко вычислить: Время полета до попадания, расстояние, которое пролетел снаряд, его конечную скорость ( по отношению к начальной, т.е. дробь меньше единицы) и конечных угол. Все эти данные в виде таблиц выгружаются в файлы csv.

\section{ Обработка результатов расчетов}

Результаты расчетов дают все необходимые данные о попадании, но не отвечают на вопрос: какой модификатор брать и попали ли мы в палубу и борт ( ну т.е. вааще ни на что не отвечают). Для этого во-первых были предложены еще две модельки ( похожие друг на друга) и сделать excel файл, чтоб это все быстро считать, а главное - подгонять.

\subsection{Модели}

Нашим целевым числом является модификатор к попаданию в тот момент, когда кидаешь кубы "сколько попало". Предлагается следующая функция
\begin{equation}
	M = \alpha \cdot T^\nu
\end{equation}
Где $M$ -- модификатор, $T$ - время полета снаряда до точки попадания, $\alpha,\nu$ - коэффициенты подгона. Для упрощения один из коэффициентов остается свободным ( его можно "подкручивать") а второй закрепляем дополнительным условием: мы знаем модификатор( задаем его) при определенных параметрах выстрела. Например для стрельбы их 8 Дюймовой пушки на 80 кабельтовых по горизонтали модификатор будет -10, а  при стрельбе на 1 кабельтов ( в упор) -4. Тогда при известной степени ( которую мы сами подбираем) решается простое уравнение
\begin{equation}
	(-4+10)= \alpha \cdot T(\text{80 Кабельтовых})^\nu
\end{equation}
Отсюда мы находим коэффициент $\alpha$ и уже можем рассчитать результирующий модификатор по формуле $-4-M$ для любого времени, только еще округлить надо.

Для бронепробиваемости идея та же самая, однако в степень возводится не время полета, а конечная скорость. Также берется бронепробиваемость, но не в конце, а на выходе из ствола степень берется отрицательная ( чтоб бропепробиваемосять падала) и рассчитывается константа. После этого используя конечную скорость можно рассчитать бронепробиваемость. Для расчета места попадания (палуба. борт и т.д.) просто берется конечный угол и выбираются его предельные значения ( если меньше  одного - то угол слишком малый и снаряд падает сверху вниз , в палубу. Если больше другого - то снизу вверх, и попадаем в дно, во всех остальных случая в борт.)
\subsection{Файл}

Для всех этих расчетов есть excel файл, в котором есть вкладка с "исходными данными", где задаются все константы, степени и исходные модификаторы. Далее там есть страницы, куда надо вставлять данные из расчетных файлов csv, и страницы с результатами расчета тех или иных нужных параметров.
\end{document}

  
  
